\newcommand{\TeamNo}{31}

\newcommand{\HWno}{04}

\newcommand{\AuthorOneName}{Merve Nur Öztürk}
\newcommand{\AuthorOneID}{2311322}

\newcommand{\AuthorTwoName}{Atakan Süslü}
\newcommand{\AuthorTwoID}{2311371}

\newcommand{\AuthorThreeName}{Betül Rana Kuran}
\newcommand{\AuthorThreeID}{2311173}


\documentclass[letterpaper,12pt]{article}
\usepackage{tabularx} % extra features for tabular environment
\usepackage{amsmath}  % improve math presentation
\usepackage{amssymb}
\usepackage{xcolor}
\usepackage{float}
\usepackage[export]{adjustbox}
\usepackage{graphicx} % takes care of graphic including machinery
\usepackage[margin=1in,letterpaper]{geometry} % decreases margins
\usepackage{cite} % takes care of citations

\begin{document}
\begin{center}
AE 305, 2020-21 Fall \hfill \textbf{HW \HWno} \hfill \textbf{Team \TeamNo} \\
\noindent\rule{\textwidth}{0.4pt}
\begin{tabular}{p{0.33\textwidth} | p{0.33\textwidth} | p{0.33\textwidth} }
	\AuthorOneName&\AuthorTwoName&\AuthorThreeName\\
	\textit{\AuthorOneID}&\textit{\AuthorTwoID}&\textit{\AuthorThreeID}
\end{tabular}
\noindent\rule{\textwidth}{0.4pt}
\end{center}

%Report start

\section{Introduction}
Finite Difference Method uses finite difference relations which are obtained from Taylor
series expansions in order to approximate the partial derivatives in a partial differential
equation. The equations representing the partial differentials are called Finite Difference
Equations. In this homework, given a convection diffusion wave equation:

\begin{equation}
	\frac{\partial \omega}{\partial t} + V\frac{\partial \omega}{\partial x} = \nu\frac{\partial^2 \omega}{\partial x^2}
	\label{eqn:gaussian}
\end{equation}

where V is the constant convection velocity, and $\nu$ is the diffusion (viscosity) coefficient,
and $\omega$ is distributed initially as follows:
\begin{equation}
	\omega(x,t=0) = exp(-b ln2(x/\Delta x)^2)
\end{equation}

where $\Delta x = 0.1$, and $b = 0.025$, it is requested to solve Equation \ref{eqn:gaussian} by modifying the given fortran code and
using forward-time central-space FDE, and prove that the equation is conditionally stable by
using different $\sigma$ (Courant number) and d values where $\sigma$ and d values are as the following:

\begin{eqnarray}
	\sigma &=& \frac{V\Delta t}{\Delta x} \nonumber \\
	d &=& \frac{\nu\Delta t}{\Delta x^2}
\end{eqnarray}
It is also asked to solve both Equation \ref{eqn:gaussian} and convection equation, where d equals
zero, by forward-time backward-space FDE and compare the solution. Afterwards, given a high order
accuracy FDE, first it is requested to perform consistency analysis on it and then solving
Equation \ref{eqn:gaussian} by using this FDE:

\begin{eqnarray}
\frac{-\omega^{n+1}_{i}+4\omega^{n}_{i}-3\omega^{n-1}_{i}}{2\Delta t} + V\frac{\omega^{n}_{i+1}-\omega^{n}_{i-1}}{2\Delta x} = \nu\frac{\omega^{n}_{i+1}-2\omega^{n}_{i}+\omega^{n}_{i-1}}{\Delta x^2}
\end{eqnarray}

Finally, implicit FDE (backward-time central-space) is used to solve Equation \ref{eqn:gaussian}.
In all solutions the range of x is such that -L/2$\leq$x$\leq$L/2 where L=40.


\section{Method}
\subsection{Finite Difference Method}
Finite Difference method is used to solve partial differential equations by 
approximating partial derivatives in these equations with finite difference relations.
In this homework, 1-D, linear, convection diffusion equation and convection equation are solved
with combination of finite difference approximations.
\begin{equation}
	\frac{\partial w}{\partial t} + V\frac{\partial w}{\partial x}=\nu\frac{\partial^2 w}{\partial x^2}
	\label{eqn:cde}
\end{equation}
is convective diffusion equation and
\begin{equation}
	\frac{\partial w}{\partial t} + V\frac{\partial w}{\partial x}=0 
	\label{eqn:ce}
\end{equation}
is convective equation.\\
Two different derivations is used for the explicit finite difference equation of 
Equation \ref{eqn:cde} in this homework.The first derivation of FDE is also used for the explicit 
FDE of Equation \ref{eqn:ce}. In both derivations, the first term of
the equation forward time approximation is used.
\begin{equation}
	\frac{\partial w}{\partial t}\vert^{n}=\frac{w^{n+1}-{w^n}}{\Delta t}+O(\Delta t)
	\label{eqn:ft}
\end{equation}
In the first derivation, for second and third terms of the Equation \ref{eqn:cde}, 
central space approximations of
$\frac{\partial w}{\partial x}\vert_{i}$ of order ($\Delta x^2$) and 
$\frac{\partial^2 w}{\partial x^2}\vert_{i}$ of order ($\Delta x^2$), respectively, are used.
\begin{equation}
	\frac{\partial w}{\partial x}\vert_{i}=\frac{w_{i+1}-w_{i-1}}{2\Delta x}+O(\Delta x^2)
	\label{eqn:cs1}
\end{equation}
\begin{equation}
	\frac{\partial^2 w}{\partial x^2}\vert_{i}=\frac{w_{i+1}-2w_i+w_{i-1}}{\Delta x^2}+O(\Delta x^2)
	\label{eqn:cs2}
\end{equation}
By substituting Equation \ref{eqn:ft}, \ref{eqn:cs1}, and \ref{eqn:cs2} into Equation \ref{eqn:cde}, it becomes:
\begin{equation}
	\frac{w_{i}^{n+1}-{w_{i}^{n}}}{\Delta t}+\frac{w_{i+1}^{n}-w_{i-1}^{n}}{2\Delta x}
	=\frac{w_{i+1}^{n}-2w_{i}^{n}+w_{i-1}^{n}}{\Delta x^2}+O(\Delta x^2, \Delta t)
\end{equation}
The explicit FDE of Equation \ref{eqn:cde} based on forward time and central spatial differences is obtained:
\begin{equation}
	w_{i}^{n+1}= w_{i}^{n}-\frac{\sigma}{2}(w_{i+1}^{n}-w_{i-1}^{n})+d(w_{i+1}^{n}-2w_{i}^{n}+w_{i-1}^{n})+O(\Delta x^2,\Delta t)
\end{equation}
where $\sigma = \frac{V\Delta t}{\Delta x }$ and $d = \frac{\nu\Delta t}{\Delta x^2 }$.
\\To derive the explicit FDE of Equation \ref{eqn:ce}, Equation \ref{eqn:ft} and \ref{eqn:cs1} are
substituted into Equation \ref{eqn:ce}
\begin{equation}
	\frac{w_{i}^{n+1}-{w_{i}^{n}}}{\Delta t}+\frac{w_{i+1}^{n}-w_{i-1}^{n}}{2\Delta x}
	=0+O(\Delta x^2, \Delta t)
\end{equation}
The explicit FDE of Equation \ref{eqn:ce} based on forward time and central spatial differences is obtained:
\begin{equation}
	w_{i}^{n+1}= w_{i}^{n}-\frac{\sigma}{2}w_{i+1}^{n}-w_{i-1}^{n}+O(\Delta x^2, \Delta t)
\end{equation}
where $\sigma = \frac{V\Delta t}{\Delta x }$.
\\In the second derivation, instead of central space approximation, backward space approximation is used.
Backward space approximation of $\frac{\partial w}{\partial x}\vert$ of order ($\Delta x$)
\begin{equation}
	\frac{\partial w}{\partial x}\vert_{i}=\frac{w_{i}-w_{i-1}}{\Delta x}+O(\Delta x)
	\label{eqn:bs1}
\end{equation}
Backward space approximation of $\frac{\partial^2 w}{\partial x^2}\vert$ of order ($\Delta x$)
\begin{equation}
	\frac{\partial^2 w}{\partial x^2}\vert_{i}=\frac{w_{i}-2w_{i-1}+w_{i-2}}{\Delta x^2}+O(\Delta x)
	\label{eqn:bs2}
\end{equation}
By substituting Equation \ref{eqn:ft}, \ref{eqn:bs1}, and \ref{eqn:bs2} into Equation \ref{eqn:cde}
\begin{equation}
	\frac{w_{i}^{n+1}-{w_{i}^{n}}}{\Delta t}+\frac{w_{i}^{n}-w_{i-1}^{n}}{\Delta x}
	=\frac{w_{i}^{n}-2w_{i-1}^{n}+w_{i-2}^{n}}{\Delta x^2}+O(\Delta x,\Delta t)
\end{equation}
The explicit FDE of Equation \ref{eqn:cde} based on forward time and backward spatial
differences is obtained:
\begin{equation}
	w_{i}^{n+1}= w_{i}^{n}-\sigma(w_{i+1}^{n}-w_{i-1}^{n})+d(w_{i}^{n}-2w_{i-1}^{n}+w_{i-2}^{n})+O(\Delta x,\Delta t)
\end{equation}
where $\sigma = \frac{V\Delta t}{\Delta x }$ and $d = \frac{\nu\Delta t}{\Delta x^2 }$.
\\ To obtain the implicit FDE of Equation \ref{eqn:cde}, backward time central space method is used.
Backward time difference is used for the first term of Equation \ref{eqn:cde}
\begin{equation}
	\frac{\partial w}{\partial t}\vert^{n}=\frac{w^{n}-w^{n-1}}{\Delta x}+O(\Delta t)
	\label{eqn:bt}
\end{equation}
Equation \ref{eqn:cs1} and \ref{eqn:cs2} are used to approximate the second and third
terms of Equation \ref{eqn:cde}, respectively. Equation \ref{eqn:bt}, \ref{eqn:cs1}
and \ref{eqn:cs2} are inserted in the Equation \ref{eqn:cde}
\begin{equation}
	\frac{w_{i}^{n}-{w_{i}^{n-1}}}{\Delta t}+\frac{w_{i+1}^{n-1}-w_{i-1}^{n-1}}{2\Delta x}
	=\frac{w_{i+1}^{n-1}-2w_{i}^{n-1}+w_{i-1}^{n-1}}{\Delta x^2}+O(\Delta x^2, \Delta t)
\end{equation}
To define a PDE as implicit, it should be discretized at time level $n+1$. So, equation becomes
\begin{equation}
	\frac{w_{i}^{n+1}-{w_{i}^{n}}}{\Delta t}+\frac{w_{i+1}^{n+1}-w_{i-1}^{n+1}}{2\Delta x}
	=\frac{w_{i+1}^{n+1}-2w_{i}^{n+1}+w_{i-1}^{n+1}}{\Delta x^2}+O(\Delta x^2, \Delta t)	
\end{equation}
The implicit FDE of Equation \ref{eqn:cde} based on BTCS Euler Implicit method is obtained:
\begin{equation}
	w_{i}^{n}= (\frac{\sigma}{2}-d)w_{i+1}^{n+1}+(2d+1)w_{i}^{n+1}-(\frac{\sigma}{2}-d)w_{i-1}^{n+1})+O(\Delta x^2,\Delta t)	
\end{equation}
where $\sigma = \frac{V\Delta t}{\Delta x }$ and $d = \frac{\nu\Delta t}{\Delta x^2 }$.

\section{Results and Discussion}

\section{Conclusion}

\end{document}