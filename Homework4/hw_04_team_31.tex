\newcommand{\TeamNo}{31}

\newcommand{\HWno}{04}

\newcommand{\AuthorOneName}{Merve Nur Öztürk}
\newcommand{\AuthorOneID}{2311322}

\newcommand{\AuthorTwoName}{Atakan Süslü}
\newcommand{\AuthorTwoID}{2311371}

\newcommand{\AuthorThreeName}{Betül Rana Kuran}
\newcommand{\AuthorThreeID}{2311173}


\documentclass[letterpaper,12pt]{article}
\usepackage{tabularx} % extra features for tabular environment
\usepackage{amsmath}  % improve math presentation
\usepackage{amssymb}
\usepackage{xcolor}
\usepackage{float}
\usepackage[export]{adjustbox}
\usepackage{graphicx} % takes care of graphic including machinery
\usepackage[margin=1in,letterpaper]{geometry} % decreases margins
\usepackage{cite} % takes care of citations

\begin{document}
\begin{center}
AE 305, 2020-21 Fall \hfill \textbf{HW \HWno} \hfill \textbf{Team \TeamNo} \\
\noindent\rule{\textwidth}{0.4pt}
\begin{tabular}{p{0.33\textwidth} | p{0.33\textwidth} | p{0.33\textwidth} }
	\AuthorOneName&\AuthorTwoName&\AuthorThreeName\\
	\textit{\AuthorOneID}&\textit{\AuthorTwoID}&\textit{\AuthorThreeID}
\end{tabular}
\noindent\rule{\textwidth}{0.4pt}
\end{center}

%Report start

\section{Introduction}
Finite Difference Method uses finite difference relations which are obtained from Taylor
series expansions in order to approximate the partial derivatives in a partial differential
equation. The equations representing the partial differentials are called Finite Difference
Equations. In this homework, given a convection diffusion wave equation:

\begin{equation}
	\frac{\partial \omega}{\partial t} + V\frac{\partial \omega}{\partial x} = \nu\frac{\partial^2 \omega}{\partial x^2}
	\label{eqn:gaussian}
\end{equation}

where V is the constant convection velocity, and $\nu$ is the diffusion (viscosity) coefficient,
and $\omega$ is distributed initially as follows:
\begin{equation}
	\omega(x,t=0) = exp(-b ln2(x/\Delta x)^2)
\end{equation}

where $\Delta x = 0.1$, and $b = 0.025$, it is requested to solve Equation \ref{eqn:gaussian} by modifying the given fortran code and
using forward-time central-space FDE, and prove that the equation is conditionally stable by
using different $\sigma$ (Courant number) and d values where $\sigma$ and d values are as the following:

\begin{eqnarray}
	\sigma &=& \frac{V\Delta t}{\Delta x} \nonumber \\
	d &=& \frac{\nu\Delta t}{\Delta x^2}
\end{eqnarray}
It is also asked to solve both Equation \ref{eqn:gaussian} and convection equation, where d equals
zero, by forward-time backward-space FDE and compare the solution. Afterwards, given a high order
accuracy FDE, first it is requested to perform consistency analysis on it and then solving
Equation \ref{eqn:gaussian} by using this FDE:

\begin{eqnarray}
\frac{-\omega^{n+1}_{i}+4\omega^{n}_{i}-3\omega^{n-1}_{i}}{2\Delta t} + V\frac{\omega^{n}_{i+1}-\omega^{n}_{i-1}}{2\Delta x} = \nu\frac{\omega^{n}_{i+1}-2\omega^{n}_{i}+\omega^{n}_{i-1}}{\Delta x^2}
\end{eqnarray}

Finally, implicit FDE (backward-time central-space) is used to solve Equation \ref{eqn:gaussian}.
In all solutions the range of x is such that -L/2$\leq$x$\leq$L/2 where L=40.


\section{Method}

\section{Results and Discussion}

\section{Conclusion}

\end{document}