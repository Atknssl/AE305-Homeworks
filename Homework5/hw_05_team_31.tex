\newcommand{\TeamNo}{31}

\newcommand{\HWno}{05}

\newcommand{\AuthorOneName}{Merve Nur Öztürk}
\newcommand{\AuthorOneID}{2311322}

\newcommand{\AuthorTwoName}{Atakan Süslü}
\newcommand{\AuthorTwoID}{2311371}

\newcommand{\AuthorThreeName}{Betül Rana Kuran}
\newcommand{\AuthorThreeID}{2311173}


\documentclass[letterpaper,12pt]{article}
\usepackage{tabularx} % extra features for tabular environment
\usepackage{amsmath}  % improve math presentation
\usepackage{amssymb}
\usepackage{xcolor}
\usepackage{float}
\usepackage[export]{adjustbox}
\usepackage{graphicx} % takes care of graphic including machinery
\usepackage[margin=1in,letterpaper]{geometry} % decreases margins
\usepackage{cite} % takes care of citations

\begin{document}
\begin{center}
AE 305, 2020-21 Fall \hfill \textbf{HW \HWno} \hfill \textbf{Team \TeamNo} \\
\noindent\rule{\textwidth}{0.4pt}
\begin{tabular}{p{0.33\textwidth} | p{0.33\textwidth} | p{0.33\textwidth} }
	\AuthorOneName&\AuthorTwoName&\AuthorThreeName\\
	\textit{\AuthorOneID}&\textit{\AuthorTwoID}&\textit{\AuthorThreeID}
\end{tabular}
\noindent\rule{\textwidth}{0.4pt}
\end{center}

%Report start

\section{Introduction}
Elliptic partial differential equations can be solved by using finite difference equations,
and the most common FDE for the solution of an elliptic PDE is obtained by second-order
central difference difference approximations of the derivatives. Afterwards, the FDE can
be solved either by direct solution methods or iterative methods. In this homework, a 2D
heat equation is given:

\begin{equation}
	\frac{\partial^2T}{\partial x^2} + \frac{\partial^2T}{\partial y^2} = 0
\end{equation}

It is requested to compute the steady state temperature distribution on a given 2D model
of a room by solving the above equation for two different cases, one is with a radiator and
the other is without a radiator. First, the iterative methods are used; Point 
Jacobi, Gauss-Seidel, SOR and Line Gauss-Seidel methods, then the solution is obtained
by the direct solution methods.

It is also asked to compare the convergence rates of the iterative methods.
\section{Method}
\section{Results and Discussion}
\section{Conclusion}
\end{document}