% EDIT HERE
% Enter team number below ??
\newcommand{\TeamNo}{31}
% Enter HW number below ??
\newcommand{\HWno}{??}
% 1st student information
\newcommand{\AuthorOneName}{Merve Nur Öztürk}
\newcommand{\AuthorOneID}{2311322}
% 2nd student information (leave empty if none)
\newcommand{\AuthorTwoName}{Atakan Süslü}
\newcommand{\AuthorTwoID}{2311371}
% 3rd student information (leavIDnumber1e empty if none)
\newcommand{\AuthorThreeName}{Betül Rana Kuran}
\newcommand{\AuthorThreeID}{2311173}
% END EDITING
% DO NOT MODIFY BELOW EXCEPT FOR ADDING PACKAGES #######
\documentclass[letterpaper,12pt]{article}
\usepackage{tabularx} % extra features for tabular environment
\usepackage{amsmath}  % improve math presentation
\usepackage{amssymb}
\usepackage{xcolor}
\usepackage{float}
\usepackage[export]{adjustbox}
\usepackage{graphicx} % takes care of graphic including machinery
\usepackage[margin=1in,letterpaper]{geometry} % decreases margins
\usepackage{cite} % takes care of citations

\begin{document}
\begin{center}
AE 305, 2020-21 Fall \hfill \textbf{HW \HWno} \hfill \textbf{Team \TeamNo} \\
\noindent\rule{\textwidth}{0.4pt}
\begin{tabular}{p{0.33\textwidth} | p{0.33\textwidth} | p{0.33\textwidth} }
	\AuthorOneName&\AuthorTwoName&\AuthorThreeName\\
	\textit{\AuthorOneID}&\textit{\AuthorTwoID}&\textit{\AuthorThreeID}
\end{tabular}
\noindent\rule{\textwidth}{0.4pt}
\end{center}

%Report start

\section{Introduction}

It is useful to use numerical methods in order to solve differential equations when
the analytical solution is time consuming. In this homework, we are given an aircraft
which rolls on the ground and takes off some time later, and we are asked to find
the minimum time required for this aircraft to take off at different airport altitudes.
We know that the aircraft will take off when the lift force is higher than the weight of
the aircraft. Since the lift force is a function of velocity;

\begin{equation}
        L = C_L \cdot \frac{1}{2} \cdot \rho_{\infty} \cdot V_{\infty}^{2}
\end{equation}

first we should calculate velocity which is described as a differential equation for
this problem:

\begin{equation}
        \frac{W}{g} \cdot \frac{dV}{dt} = T - D - \mu \cdot (W - L)
\end{equation}

Therefore we are asked to use Euler's and RK2 methods so that we can approach the
solution with the help of numerical methods.

(Drag formülünü de yazmak istedim ama nereye yazarım bilemedim sonra eklerim)

\section{Method}

\newpage

\section{Results and Discussion}

\subsection{Subsection}

\section{Conclusion}

\end{document}