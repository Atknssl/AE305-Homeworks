\newcommand{\TeamNo}{31}

\newcommand{\HWno}{03}

\newcommand{\AuthorOneName}{Merve Nur Öztürk}
\newcommand{\AuthorOneID}{2311322}

\newcommand{\AuthorTwoName}{Atakan Süslü}
\newcommand{\AuthorTwoID}{2311371}

\newcommand{\AuthorThreeName}{Betül Rana Kuran}
\newcommand{\AuthorThreeID}{2311173}


\documentclass[letterpaper,12pt]{article}
\usepackage{tabularx} % extra features for tabular environment
\usepackage{amsmath}  % improve math presentation
\usepackage{amssymb}
\usepackage{xcolor}
\usepackage{float}
\usepackage[export]{adjustbox}
\usepackage{graphicx} % takes care of graphic including machinery
\usepackage[margin=1in,letterpaper]{geometry} % decreases margins
\usepackage{cite} % takes care of citations

\begin{document}
\begin{center}
AE 305, 2020-21 Fall \hfill \textbf{HW \HWno} \hfill \textbf{Team \TeamNo} \\
\noindent\rule{\textwidth}{0.4pt}
\begin{tabular}{p{0.33\textwidth} | p{0.33\textwidth} | p{0.33\textwidth} }
	\AuthorOneName&\AuthorTwoName&\AuthorThreeName\\
	\textit{\AuthorOneID}&\textit{\AuthorTwoID}&\textit{\AuthorThreeID}
\end{tabular}
\noindent\rule{\textwidth}{0.4pt}
\end{center}

%Report start

\section{Introduction}
\label{section:intro}

Finite Volume Method is a numerical method used in order to solve partial differential
equations representing the conservation of a quantity. In this homework, potential flow
field over a circular cylinder at various values of angle of attack and a symmetric NACA
airfoil at zero angle of attack is requested to be calculated. For a potential flow,
velocity field is obtained by the Laplace's equation:

\begin{equation}
	\vec{\nabla} \cdot \vec{\nabla}\phi = \frac{\partial^2 \phi}{\partial x^2} + \frac{\partial^2 \phi}{\partial y^2} = 0.
\end{equation}

\begin{equation}
\vec{V} = \vec{\nabla}\phi
\end{equation}

A pseudo time derivative is introduced in order to solve this equation without a system
of linear algebraic equations.

\begin{equation}
	\frac{\partial \phi}{\partial t} = \nu \left(\frac{\partial^2 \phi}{\partial x^2} + \frac{\partial^2 \phi}{\partial y^2}\right)
\end{equation}

where $\nu$ is an artificial diffusion coefficient. Steady state boundary conditions are
set to calculate the steady state solution as the time derivative goes to zero during the
integration. The integral form of the above equation is:

\begin{equation}
	\frac{\partial}{\partial t} \int_{\Omega}\phi  d\Omega + \oint_{S} \vec{F} \cdot d\vec{S} = 0
\end{equation}

\begin{equation}
	\vec{F} = -\nu\frac{\partial \phi}{\partial x}\vec{i} - \nu\frac{\partial \phi}{\partial y}\vec{j}
\end{equation}

Two boundary conditions are set such that the velocity is equal to the free stream velocity
far away from the body (Farfield BC) and there is no flow normal to the body, which is no
penetration boundary condition (Wall BC):

\begin{eqnarray}
	\vec{\nabla}\phi|_{far BC} &=& \vec{V}_{\infty} = \vec{u}_{\infty}\vec{i} + \vec{v}_{\infty}\vec{j} = V_{\infty}(cos\alpha\vec{i}+sin\alpha\vec{j})\\
	&=& V_{\infty}cos\alpha\vec{i} + V_{\infty}sin\alpha\vec{j}
\end{eqnarray}

\begin{equation}
	(\vec{n}\cdot\vec{V}_{\infty})_{wall} = 0 or (\vec{F}\cdot\vec{S})_{wall} = 0.
\end{equation}

The conserved quantity for this problem is the potential $\phi$, and the FV method is applied
for the calculation of this quantity as explained above.



\section{Method}
\subsection{Finite Volume Method} 
In this homework, Finite Volume Method is used. In this method, an integral conservation law
\begin{equation}
        \frac{\partial}{\partial t}\int_{\Omega_e} q\,d\Omega + \oint_{S_e} \vec{F} \cdot \vec{dS} =0     
        \label{eqn:fvm}
\end{equation}
is applied to each control volume $\Omega_e$, and its boundries $S_e$ in a discrete form. First, solution domain is divided into small triangular cells and a computational mesh is created. 
Then, all cells are labeled and neighbour cells to their surfaces are identified. Then, conservation law is applied to all cells.\\
A pseudo time derivative is substituted into Laplace equation which is multiplied by artificial 
diffusion coefficient,$\nu$. The equation becomes
\begin{equation}
        \frac{\partial \Phi}{\partial t} - \nu \vec{\nabla}\cdot\vec{\nabla}\Phi = 0
        \label{eqn:pseudodiff}
\end{equation}
where $\vec{F}=-\nu\vec{\nabla}\Phi$\\
Since a steady state solution is obtained as the derivative goes to zero and steady boundry conditions 
are specified,Laplace’s equation is satisfied. 
The integral form of Equation \ref{eqn:pseudodiff} becomes
\begin{eqnarray}
        \frac{\partial}{\partial t}\int_{\Omega} \Phi\,d\Omega + \oint_S \vec{F} \cdot \vec{dS} =0\ \\
        \mbox{where}\: \vec{F}=f\vec{i}+g\vec{j}=-(\nu\frac{\partial \Phi}{x}\vec{i}+\nu\frac{\partial \Phi}{y}\vec{j})\nonumber
        \label{eqn:genel}
\end{eqnarray}
Equation \ref{eqn:fvm} and Equation \ref{eqn:genel} have the same form. Therefore, finite volume method can be applied.\\
A simple discete form of the Equation \ref{eqn:genel} applied to a two dimensional triangular cell $e$ is then given by
\begin{equation}
        \frac{d(\Phi_e\Omega_e)}{t}+\sum_{s=1}^{3} (\vec{F} \cdot \vec{dS})\ =0
        \label{eqn:diff}
\end{equation}
where $\Phi_e$ is the average $\Phi$ over element e, $\Omega_e$ is the area of triangle $e$ ,and
$\vec{S} = \Delta y_{e,s}\vec{i}-\Delta x_{e,s}\vec{j}$\\
In this homework, averaging of cell-based quantities approach is used to calculate flux values. According to this approach,
it should be taken that averages of the cell-based flux values of the cells sharing the same surface to compute the flux 
through this surface.Therefore, the second term of \ref{eqn:diff} becomes
\begin{eqnarray}
	\sum_{s=1}^{3} (\vec{F} \cdot \vec{dS})&=&\sum_{s=1}^{3} (\vec{F_{e,s}} \cdot \vec{dS_{e,s}}) \nonumber \\
	&=&\sum_{s=1}^{3}[\frac{1}{2}(f_{e}+f_{e,ns})\Delta y_{e,s}-\frac{1}{2}(g_{e}+g_{e,ns})\Delta x_{e,s}] 
\end{eqnarray}

Equation \ref{eqn:diff} for element e becomes
\begin{equation}
        \Omega_e \frac{d\Phi_e}{dt} +\sum_{s=1}^{3} [f_{e,s}\Delta y_{e,s}-g_{e,s}\Delta x_{e,s}]=0
\end{equation}
By using Taylor Series Expansion in time, time derivative is approxiamted. These steps leads to the explicit Euler time
integration:
\begin{equation}
        \Phi_{e}^{k+1} =\Phi_{e}^{k} - \frac{\Delta t}{\Omega_e}\sum_{s=1}^{3}[f_{e,s}^{k}\Delta y_{e,s}-g_{e,s}^{k}\Delta x_{e,s}]
\end{equation}
Since $\vec{F}$ is diffusive flux, average gradient vectors are calculated by
\begin{eqnarray}
\bar{\frac{\partial \Phi}{\partial x}}\bigg|_{e}&=&\frac{1}{\Omega_e}\sum_{s=1}^{3}\frac{1}{2}(\Phi_e+\Phi{ns})\Delta y_{e,s} \nonumber \\
\bar{\frac{\partial \Phi}{\partial y}}\bigg|_{e}&=&-\frac{1}{\Omega_e}\sum_{s=1}^{3}\frac{1}{2}(\Phi_e+\Phi{ns})\Delta x_{e,s}	\nonumber
\end{eqnarray}
\newpage

\section{Results and Discussion}

\subsection{Flowfield Around A Cylinder}

\begin{figure} [ht]
	\centering
	\includegraphics[height = 10cm]{graph/15deg/Cylinder_15angle_speed0001.png}
	\caption{Potential flow over a circular cylinder at $\alpha=15$ calculated by FV method.}
    \label{fig:q1p}
\end{figure}

According to the boundary conditions mentioned in Section \ref{section:intro}, FV fortran code
is completed. Afterwards, flowfield around a cylinder is calculated using the unstructured grid provided.

\newpage

\subsection{Flowfield Around A Cylinder with Different Angle of Attacks}
\begin{figure} [ht]
	\centering
	\includegraphics[height = 9.5cm]{graph/0deg/Cylinder_0angle_streamline0000.png}
	\caption{Streamlines over a circular cylinder at $\alpha=0$ calculated by FV method.}
    \label{fig:q2st0}
\end{figure}

In Figure \ref{fig:q2st0},\ref{fig:q2st15},and \ref{fig:q2st-15}, it can be observed that streamlines 
are parallel and horizontal lines far from the cylinder, and they are in the direction of the free stream velocity,
which results from the Farfield boundary condition. However, as approached the cylinder, streamlines are squeezed,
and do not across on each other and through the cylinder, since the normal component of velocity at 
the surface of cylinder is equal to zero. This is a consequence of the no penetration boundary condition, which
is also called Wall boundary condition.

\newpage

\begin{figure} [!h]
	\centering
	\includegraphics[height = 9.5cm]{graph/15deg/Cylinder_15angle_streamline0000.png}
	\caption{Streamlines over a circular cylinder at $\alpha=15$ calculated by FV method.}
    \label{fig:q2st15}
\end{figure}

\begin{figure} [!h]
	\centering
	\includegraphics[height = 9.5cm]{graph/neg15deg/Cylinder_neg15angle_streamline0000.png}
	\caption{Streamlines over a circular cylinder at $\alpha=-15$ calculated by FV method.}
    \label{fig:q2st-15}
\end{figure}

\newpage

\begin{figure} [!h]
	\centering
	\includegraphics[height = 9.5cm]{graph/0deg/Cylinder_0angle_vector0000.png}
	\caption{The velocity vectors over a circular cylinder at $\alpha=0$ calculated by FV method.}
    \label{fig:q2v0}
\end{figure}

\begin{figure} [!h]
	\centering
	\includegraphics[height = 9.5cm]{graph/15deg/Cylinder_15angle_vector0000.png}
	\caption{The velocity vectors over a circular cylinder at $\alpha=15$ calculated by FV method.}
    \label{fig:q2v15}
\end{figure}

\newpage

\begin{figure} [ht]
	\centering
	\includegraphics[height = 9.5cm]{graph/neg15deg/Cylinder_neg15angle_vector0000.png}
	\caption{The velocity vectors over a circular cylinder at $\alpha=-15$ calculated by FV method.}
    \label{fig:q2v-15}
\end{figure}
According to Figure \ref{fig:q2v0},\ref{fig:q2v15},and \ref{fig:q2v-15}, velocity vectors have the 
same direction, and magnitudes ($\vec{V}_{\infty}$) far from cylinder, which satisfies the Farfield boundry condition.
Also, the velocity vectors near the cylinder is tangent to the boundary. On the front and back of the 
cylinder, there are two points where the direction of the velocity vector is too close to normal
of surface. However, the magnitude of velocity is so small that they can be considered as zero,
satisfying the Wall boundary condition. These points are called stagnation points.

\newpage

\subsection{NACA0012 airfoil mesh}
\label{sec:medium}

\begin{figure} [!h]
	\centering
	\includegraphics[height = 9.5cm]{graph/medium/medium_62650000.png}
	\caption{Unstructured mesh of NACA0012 airfoil}
    \label{fig:airfoilmesh}
\end{figure}

\vspace{2cm}

Unstructured mesh of NACA0012 airfoil is generated using Easymesh software. This 
unstructured mesh consists of 6265 triangular grids.

\newpage

\begin{figure} [!h]
	\centering
	\includegraphics[height = 9.5cm]{graph/medium/medium_62650001.png}
	\caption{Unstructured mesh of NACA0012 airfoil near the airfoil}
    \label{fig:airfoilmeshclose}
\end{figure}

\vspace{2cm}

It can be seen from Figure \ref{fig:airfoilmesh} and Figure \ref{fig:airfoilmeshclose} that, grids around 
the airfoil are smaller for better accuracy, whereas grid far from the airfoil are relatively bigger.
\newpage

\subsection{Pressure distribution around NACA0012 airfoil at $\alpha = 0$}

\begin{figure} [!h]
	\centering
	\includegraphics[height = 9.5cm]{graph/medium/medium_streamline0000.png}
	\caption{Streamlines around NACA0012 airfoil}
    \label{fig:airfoilstreamline}
\end{figure}

When potential flow equation is solved using Finite Volume method using unstructured mesh in Figure \ref{fig:airfoilmesh},
flow around NACA0012 airfoil can be found. This flow can be illustrated using streamlines, as can be seen in 
Figure \ref{fig:airfoilstreamline}. As expected, air flows faster at the airfoil camber, and air stagnates at the 
leading edge and the trailing edge. Moreover, pressure coefficient can be calculated using Bernoulli's Equation 
as the speed of the flow is known.

\newpage

\begin{figure} [!h]
	\centering
	\includegraphics[height = 9.5cm]{graph/medium/medium_pressure0000.png}
	\caption{Pressure coefficient around NACA0012 airfoil}
    \label{fig:airfoilpressure}
\end{figure}

It can be seen from Figure \ref{fig:airfoilpressure} that pressure distribution on the upper and the lower surfaces are the same.
This is because airfoil is symmetric and airfoil is at zero angle of attack. If the angle of attack was different from zero, 
pressure distribution on the upper and lower surfaces would be different. It can also be seen that at leading edge and trailing edge,
pressure is relatively larger; and around maximum camber point, pressure is relatively lower. This is because air flows slower at the 
leading and the trailing edge and flows faster around maximum camber point.

\newpage

\subsection{Solutions with different grid sizes}

\subsubsection{Coarse grid}

\begin{figure} [!h]
	\centering
	\includegraphics[height = 9.5cm]{graph/coarse/coarse_8470000.png}
	\caption{Coarse unstructured mesh of NACA0012 airfoil}
    \label{fig:airfoilmeshcoarse}
\end{figure}

\vspace{2cm}

Coarse unstructured mesh of NACA0012 airfoil is generated using Easymesh software. This 
unstructured mesh consists of 847 triangular grids.

\newpage

\begin{figure} [!h]
	\centering
	\includegraphics[height = 9.5cm]{graph/coarse/coarse_8470001.png}
	\caption{Coarse unstructured mesh of NACA0012 airfoil near the airfoil}
    \label{fig:airfoilmeshcoarseclose}
\end{figure}

\begin{figure} [!h]
	\centering
	\includegraphics[height = 9.5cm]{graph/coarse/coarse_streamline0000.png}
	\caption{Streamlines around NACA0012 airfoil with coarse grid}
    \label{fig:airfoilcoarsestreamline}
\end{figure}

\newpage

It can be seen from Figure \ref{fig:airfoilcoarsestreamline} that, when coarse grid is 
use for calculation, streamlines obeys no penetration boundary condition but magnitude 
of velocity is not calculated correctly around the airfoil.

\vspace{2cm}

\begin{figure} [!h]
	\centering
	\includegraphics[height = 9.5cm]{graph/coarse/coarse_pressure0000.png}
	\caption{Pressure coefficient around NACA0012 airfoil with coarse grid}
    \label{fig:airfoilcoarsepressure}
\end{figure}

\vspace{2cm}

Around the leading and the trailing edge, pressure coefficient distribution is 
closer to the correct distribution but still not correct enough. Around the maximum
camber point, pressure distribution is way off the correct result. Therefore, coarse
mesh is not suitable for precise calculation of pressure around an airfoil.

\newpage

\subsubsection{Fine grid}

\begin{figure} [!h]
	\centering
	\includegraphics[height = 9.5cm]{graph/fine/fine_209480000.png}
	\caption{Fine unstructured mesh of NACA0012 airfoil}
    \label{fig:airfoilmeshfine}
\end{figure}

Fine unstructured mesh of NACA0012 airfoil is generated using Easymesh software. This 
unstructured mesh consists of 20948 triangular grids.

\newpage

\begin{figure} [!h]
	\centering
	\includegraphics[height = 9.5cm]{graph/fine/fine_209480001.png}
	\caption{Fine unstructured mesh of NACA0012 airfoil near the airfoil}
    \label{fig:airfoilmeshfineclose}
\end{figure}
\begin{figure} [!h]
	\centering
	\includegraphics[height = 9.5cm]{graph/fine/fine_streamline0000.png}
	\caption{Streamlines around NACA0012 airfoil with fine grid}
    \label{fig:airfoilfinestreamline}
\end{figure}

\newpage

When a fine grid is used to calculate flow around the airfoil, velocity distribution
can be found correctly around the airfoil. Figure \ref{fig:airfoilfinestreamline} 
illustrates that streamlines are calculated according to the no penetration boundary 
condition and air flows faster around the cambered part of the airfoil.

\vspace{2cm}

\begin{figure} [!h]
	\centering
	\includegraphics[height = 9.5cm]{graph/fine/fine_pressure0000.png}
	\caption{Pressure coefficient around NACA0012 airfoil with fine grid}
    \label{fig:airfoilfinepressure}
\end{figure}

\vspace{1cm}

As Figure \ref{fig:airfoilfinepressure} shows, pressure distribution can be 
calculate more precisely when a finer grid is used. But the usage of finer grid
greatly increases the computation time, so unnecessarily fine grid should be avoided 
to maximize efficiency.

\subsubsection{Medium grid}

Calculations of Section \ref{sec:medium} is done using a medium grid, which consists of 
6255 triangular grids. When the results of this section is considered, it can be seen that
medium grid is precise enough to give correct results and because of its relatively lower triangle count, result
can be calculated faster. Therefore, using a medium grid is more efficient.

\newpage

\section{Conclusion}

\end{document}